\documentclass[]{scrreprt}


% Title Page
\title{Stats under the stars$^3$}
\author{Markov Unchained \\ \small \textit{''the second lag is silent''}}


\begin{document}
\maketitle

\paragraph{Descrizione del dataset}
Il problema di previsione affrontato consisteva nel selezionare su un test set i 10mila clienti della banca Findomestic che risultano pi\`u propensi ad accettare una proposta commerciale in seguito a una telefonata. 

Il problema si qualificava come un'analisi supervisionata con un \textit{training set} (40mila obs.) che riportava il valore della variabile target, in questo caso una dummy che assumeva valore $1$ in caso il cliente chiamato avesse accolto la proposta ricevuta e $0$ in caso contrario. 

Tale \textit{training set} presentava inoltre un forte sbilanciamento per quanto riguardava la distribuzione della variabile \textit{target}: infatti ben il $95\%$ dei casi aveva valore target pari a $0$.

\paragraph{Undersampling}

Per ovviare all'inconveniente dello sbilanciamento del dataset si \`e ricorso ad una procedura di \textit{undersampling}: una volta campionate delle osservazioni con valore target $1$ si é campionato lo stesso numero di osservazioni tra quelle con target $0$. Il dataset ottenuto aveva dunque lo stesso numero di $0$ e $1$.

Oltre a risolvere il problema dello sbilanciamento, l'undersampling ci ha permesso di valutare la bont\`a di modelli diversi dato che un minor numero di osservazioni riduceva sensibilmente il peso computazionale.

\paragraph{Feature engineering}  Nella prima parte del lavoro ci siamo concentrati sulla costruzione di variabili che aiutassero a migliorare la previsione. Inizialmente abbiamo rimosso alcune variabili a causa dell'elevato numero di dati mancanti quali anzianit\`a bancaria (\verb|ANZ_BAN|) o il numero di pratiche riguardanti i prestiti personali(\verb|FIND_PPQ_C_NUM_PRA|). Per altre si \`e ricorso all'imputazione tramite media sia 

Per quanto riguarda la creazione di nuove variabili il criterio utilizzato \`e stato quello della riduzione del numero di regressori da dare in pasto ai modelli che via via abbiamo considerato.

\paragraph{Random Forest}


\end{document}          
